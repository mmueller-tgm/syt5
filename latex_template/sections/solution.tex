%!TEX root=../document.tex
\section{Code}
Der Code ist bei \href{https://github.com/mmueller-tgm/syt5}{https://github.com/mmueller-tgm/syt5} bereitgestellt.
\section{Einstieg}
Am Anfang wollte ich, durch verschiedene Tutorials selber in Java/Node den REST
 Service implementieren aber nach mehreren Problemen mit Packages wurde ich auf
 ein Tutorial\footnote{\href{https://hellokoding.com/registration-and-login-example-with-spring-security-spring-boot-spring-data-jpa-hsql-jsp/}{https://hellokoding.com/registration-and-login-example-with-spring-security-spring-boot
-spring-data-jpa-hsql-jsp/}} hingewiesen, welches mehr oder weniger die Aufgabe
 abbildet.

Voraussetzung ist JDK1.7 oder später und Maven 3 zum Builden und Deployen.
Den Sourcecode erhält man mit GIT ,,git clone git@github.com:hellokoding/
registration-login-spring-hsql.git``. 

Per default ist ,,/login`` schon als Endpunkt konfiguriert ist aber ,,/register``
 nicht. Um ,,/registration`` auf ,,/register`` umzuleiten habe ich folgendes
 Command verwendet:
\begin{lstlisting}[language=Python]
	find . -type f -exec sed -i 's/registration/register/g' {} +
\end{lstlisting}
Es sucht in jedem File in dem Ordner und tauscht ,,registration`` mit
 ,,register`` aus.
 
\section{Ausf\"uhren des Codes}
Deployed wird mit ,,mvn clean spring-boot:run``, Test cases mit ,,python3.5 tests.py``. Der Service rennt auf ,,localhost:8080``.
 
\section{Erkl\"arung der Codebase}
\subsection{pom.xml}
Da die Buildumgebung Maven ist, werden in ,,pom.xml`` die benötigten Packages angegeben.

\subsection{src/main/java/com/hellokoding/auth/model/User.java}
Hier wird die Struktur der User Tabelle bzw. das User Objekt definiert.

\subsection{src/main/java/com/hellokoding/auth/model/Role.java}
Hier wird die Struktur der Role Tabelle bzw. das Role Objekt definiert.

\subsection{Spring Security}
In ,,src/main/java/com/hellokoding/auth/service/SecurityServiceImpl.java`` werden eingeloggte User und neu erstellte User authentifiziert.

In ,,src/main/java/com/hellokoding/auth/service/UserServiceImpl.java`` wird das Service zum registrieren von Usern bereitgestellt.

\subsection{Spring Validator}  
In ,,src/main/java/com/hellokoding/auth/validator/UserValidator.java`` werden angegebene Userdaten Validiert, die bei dem Registrieren angegeben werden.

\subsection{Controller (Endpunkte)}
,,src/main/java/com/hellokoding/auth/web/UserController.java`` handeld alle POST und GET Requests. 

\begin{lstlisting}[language = Java]
package com.hellokoding.auth.web;

import com.hellokoding.auth.model.User;
import com.hellokoding.auth.service.SecurityService;
import com.hellokoding.auth.service.UserService;
import com.hellokoding.auth.validator.UserValidator;
import org.springframework.beans.factory.annotation.Autowired;
import org.springframework.stereotype.Controller;
import org.springframework.ui.Model;
import org.springframework.validation.BindingResult;
import org.springframework.web.bind.annotation.ModelAttribute;
import org.springframework.web.bind.annotation.RequestMapping;
import org.springframework.web.bind.annotation.RequestMethod;

@Controller
public class UserController {
    @Autowired
    private UserService userService;

    @Autowired
    private SecurityService securityService;

    @Autowired
    private UserValidator userValidator;

    @RequestMapping(value = "/register", method = RequestMethod.GET)
    public String register(Model model) {
        model.addAttribute("userForm", new User());

        return "register";
    }

    @RequestMapping(value = "/register", method = RequestMethod.POST)
    public String register(@ModelAttribute("userForm") User userForm, BindingResult bindingResult, Model model) {
        userValidator.validate(userForm, bindingResult);

        if (bindingResult.hasErrors()) {
            return "register";
        }

        userService.save(userForm);

        securityService.autologin(userForm.getUsername(), userForm.getPasswordConfirm());

        return "redirect:/welcome";
    }

    @RequestMapping(value = "/login", method = RequestMethod.GET)
    public String login(Model model, String error, String logout) {
        if (error != null)
            model.addAttribute("error", "Your username and password is invalid.");

        if (logout != null)
            model.addAttribute("message", "You have been logged out successfully.");

        return "login";
    }

    @RequestMapping(value = {"/", "/welcome"}, method = RequestMethod.GET)
    public String welcome(Model model) {
        return "welcome";
    }
}
\end{lstlisting}

\section{Unittests}
Unittests wurden mit den Python Bibliotheken ,,unittests`` und ,,requests`` 
durchgeführt. 

\subsubsection{Ansprechen der REST Schnittstelle}
Um mit dem Service zu kommunizieren müssen, per Definition, HTTP Requests 
gesendet werden. Dazu eignet sich besonders die Bibliothek ,,requests``.

\subsubsection{Registrieren eines Benutzers}
Um einen Benutzer zu registrieren muss zuerst ein Request zu /register gemacht 
werden, um einen Sessionkey und einen csrf-Key zu bekommen.  
\begin{lstlisting}[language=Python]
x = requests.get("%sregister"%self.server)
cookies = x.cookies
x = BeautifulSoup(x.text, 'html.parser')
x = x.find_all("input")
payload={}
for y in x:
	if y["name"] == "_csrf":
	    payload[y["name"]] = y["value"]
\end{lstlisting}

Mit diesen Keys kann man dann mit POST einen User erstellen.

\begin{lstlisting}[language=Python]
payload["username"] = "test_%s"%''.join(random.sample('0123456789',5))
payload["password"] = '123456789'
payload["passwordConfirm"] = '123456789'
response = requests.post("%sregister"%self.server, payload, cookies=id)
\end{lstlisting}

\subsubsection{test\_create\_user}
Dieser Testfall erstellt einen Benutzer und checkt ob dieser eingeloggt wird.
\begin{lstlisting}[language=Python]
user="test_%s"%''.join(random.sample('0123456789',5))
response = self.create_user(user=user)
y = BeautifulSoup(response.text, 'html.parser').find_all("form")
y2 = BeautifulSoup(response.text, 'html.parser').find_all("span")

for z in y:
    if z['id'] == 'logoutForm':
        self.assertTrue(True, "Logged in successfully")
        return
self.assertTrue(False)
\end{lstlisting}

\subsubsection{test\_create\_user\_short\_pw}
Dieser Testfall erstellt einen Benutzer mit zu kurzem Passwort und checkt, ob
 dieser eingeloggt wird, wenn ja, dann failt der Test.
\begin{lstlisting}[language=Python]
user="test_%s"%''.join(random.sample('0123456789',5))
response = self.create_user(user=user, passwd='1234567')
y = BeautifulSoup(response.text, 'html.parser').find_all("form")
y2 = BeautifulSoup(response.text, 'html.parser').find_all("span")

for z in y:
    if z['id'] == 'logoutForm':
        self.assertTrue(False, "Should not have been able to login.")
        return

for z in y2:
    if z['id'] == 'passeord.errors':
        self.assertTrue(True)
        return
    if z['id'] == 'username.errors':
        self.assertTrue(False, "Should not have username errors:")
\end{lstlisting}

\subsubsection{test\_create\_user\_short\_un}
Dieser Testfall erstellt einen Benutzer mit zu kurzem Usernamen und checkt, ob
 dieser eingeloggt wird, wenn Ja, dann failt der Test.
\begin{lstlisting}[language=Python]
user="test_"
response = self.create_user(user=user, passwd='12345678')
y = BeautifulSoup(response.text, 'html.parser').find_all("form")
y2 = BeautifulSoup(response.text, 'html.parser').find_all("span")

for z in y:
    if z['id'] == 'logoutForm':
        self.assertTrue(False, "Should not have been able to login.")
        return

for z in y2:
    if z['id'] == 'passeord.errors':
        self.assertTrue(False, "Password should be long enough.")
        return
    if z['id'] == 'username.errors':
        if z.next == "Please use between 6 and 32 characters.":
            self.assertTrue(True, "Should not have username errors:")
        else:
            self.assertTrue(False, "UN should be unique")
\end{lstlisting}

\subsubsection{test\_login}
Dieser Testfall erstellt einen Benutzer. In einer anderen Session loggt er sich
 dann ein. Bei erfolgreichem einloggen ist auch der Testfall erfolgreich. 
\begin{lstlisting}[language=Python]
user = "test_%s" % ''.join(random.sample('0123456789', 5))
passwd = "123456789"
self.create_user(user, passwd=passwd)
x = requests.get("%slogin"%self.server)
id = x.cookies
x = BeautifulSoup(x.text, 'html.parser')
x = x.find_all("input")
payload={"username":user, "password":passwd}
for y in x:
    if y["name"] == "_csrf":
        payload[y["name"]] = y["value"]

response = requests.post("%slogin"%self.server, payload, cookies=id)
response = BeautifulSoup(response.text, 'html.parser')
y = BeautifulSoup(response.text, 'html.parser').find_all("form")
for z in y:
    if z['id'] == 'logoutForm':
        self.assertTrue(True, "Logged in successfully")
        return
    self.assertTrue(False)
\end{lstlisting}

\subsubsection{test\_login\_false}
Dieser Testfall erstellt einen Benutzer. In einer anderen Session loggt er sich
 dann mit falschem Passwort ein. Bei erfolgreichem Einloggen ist der Testfall nicht erfolgreich. 
\begin{lstlisting}[language=Python]
user = "test_%s" % ''.join(random.sample('0123456789', 5))
passwd = "123456789"
self.create_user(user, passwd=passwd)
x = requests.get("%slogin"%self.server)
id = x.cookies
x = BeautifulSoup(x.text, 'html.parser')
x = x.find_all("input")
payload={"username":user, "password":passwd+"x"}
for y in x:
    if y["name"] == "_csrf":
        payload[y["name"]] = y["value"]

response = requests.post("%slogin"%self.server, payload, cookies=id)
response = BeautifulSoup(response.text, 'html.parser')
y = BeautifulSoup(response.text, 'html.parser').find_all("form")

for z in y:
    if z['id'] == 'logoutForm':
        self.assertTrue(False, "Used Wrong PW, should not have logged in")
        return
    self.assertTrue(True)
\end{lstlisting}

